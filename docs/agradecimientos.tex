\begin{center}
\parbox{1.0\linewidth}{	
	\begin{picture}(300,5)
		\put(0,-30){\includegraphics[scale=0.05,angle=0]{mandelbrot.png}}
		\put(110,0){{\huge \textbf{\textit{Agradecimientos}}}}
		\put(340,30){\includegraphics[scale=0.05,angle=180]{mandelbrot.png}}
	\end{picture}
}

\vspace{30pt}
\parbox{1.0\linewidth}{
	No me bastaría con solo dar las gracias a quienes con dedicación y paciencia cambiaron por completo mi percepción sobre las matemáticas haciendo posible en mí una comprensión más que satisfactoria de este lenguaje, y que me permitirá explorar por mi cuenta nuevos conocimientos. Mostrar que sus esfuerzos no han sido en vano al buscar nuevos retos que me sigan formando como matemático, como respuesta de agradecimiento.

\vspace{6pt}
	Entre buenas enseñanzas, consejos, y sobre todo conocimientos que considero valiosos, debo mencionar que me encuentro totalmente convencido de haber elegido el mejor lugar para formarme en muchos aspectos; Y pensar que una decisión espontánea me ha llevado hacia un lugar donde aprender matemáticas es un placer,  considerando el \textbf{\textit{Departamento de Matemáticas de la Universidad de Sonora}} un hogar para alguien que iba de pasada con el interés de solo conocer lo que allí se enseña, y que sin querer se ha quedado para integrarse en este departamento.

\vspace{6pt}	
	Un pequeño impulso solo basto, cuando un par de manchas captaron mi atención, y ya motivado e intentando aprender algo sobre binomios, en un abrir y cerrar de ojos, ya me encontraba con mi cabeza midiendo bolas, cubiertas, y entre otras cosas extrañas; Pero aun así, fue fascinante conocer estas rarezas. Esto es tan solo un poco del reflejo de satisfacción que me lleve de ese lugar, dando el mérito a cada uno de los profesores y alumnos que aportaron a mi desarrollo intelectual, y que difícilmente podre encontrar algo más emocionante que aquellas discusiones al plantear y resolver problemas con algún compañero, profesor o consigo mismo.

\vspace{6pt}
	Es una gran satisfacción finalizar este trabajo de tesis, que logro darme conocimientos considerados de gran valor, pero que sin duda, esto no hubiera sido posible sin la participación de mis profesores; \textbf{\textit{Daniel Olmos, Saúl Díaz, Martín Gildardo y Francisco Delgado}}. Por eso mi mayor agradecimiento va dirigido a ellos, que por su apoyo y disposición al ser tutores de este trabajo logramos finalizarlo con éxito. Bastante es mi agradecer por todo el conocimiento que he adquirido gracias a sus aportaciones, que lejos de poder compensarlo, mi mejor manera de responder es sacando todo su provecho posible para usarlo de manera inteligente; y que por escrito doy mi palabra que cada una de estas aportaciones no han sido actos en vano.
}

\vspace{5pt}
\begin{flushright}
	\textbf{\textit{
		Alan Matzumiya, by \textit{Circuital $\|$ Minds}\\
		Hermosillo, Sonora. 18 de septiembre del 2020\\
		}
	}
\end{flushright}

\vspace{5pt}
\begin{flushleft}
\parbox{0.45\linewidth}{
\textit{	
	"Me encontré en la posición de ese niño de un cuento que ve un trozo de cuerda y, por curiosidad, tira de ella para descubrir que era solo la punta de una cuerda muy larga y cada vez más gruesa". \textbf{ - Benoît Mandelbrot}.
	}
}
\end{flushleft}
\end{center}